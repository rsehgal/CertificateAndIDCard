\documentclass[landscape]{article}
\usepackage{wallpaper}
\usepackage{niceframe}
\usepackage{xcolor}
\usepackage{ulem}
\usepackage{graphicx}
\usepackage{geometry}
\geometry{tmargin=.5cm,bmargin=.5cm,
lmargin=.5cm,rmargin=.5cm}
\usepackage{multicol}
\setlength{\columnseprule}{0.4pt}
\columnwidth=0.3\textwidth

\begin{document}

\TileWallPaper{4cm}{2cm}{tiling.png}

\centering
\scalebox{3}{\color{green!30!black!60}
\begin{minipage}{.33\textwidth}
\font\border=umrandb
\generalframe
{\border \char113} % up left
{\border \char109} % up
{\border \char112} % up right
{\border \char108} % left 
{\border \char110} % right
{\border \char114} % lower left
{\border \char111} % bottom
{\border \char115} % lower right
{\centering

\begin{minipage}{.9\textwidth}
\centering
\includegraphics[height=1.1cm]{escudozz.pdf}
\end{minipage}
\vspace{-8mm}

\curlyframe[.9\columnwidth]{

\textcolor{red!10!black!90}
{\small University of Nobodyknows}\\

\textcolor{green!10!black!90}{
\tiny In honour of out standing performance and dedication to waste time in class we hereby award the}

\smallskip

\textcolor{red!30!black!90}
{\textit{Certificate of}}

\textcolor{black}{\large \textsc{Biggest Sleeper Class}}

\vspace{2mm}

\tiny
to: \uline{\textcolor{black}
{Mr. Dormouse Overwintering Marmot}}

(Master degree)

\vspace{4mm}

{\color{blue!40!black}
\scalebox{.7}{
\begin{tabular}{ccccc}
\cline{1-1} 
\cline{3-3}
\cline{5-5}
\\
Dr. DavidRestless  & &  Dr. Peter Awakened & & Dr. John Workerhard \\
Head of  Department & & Examinor & & Academic Advisor \\ 
\end{tabular}
}}}}
\end{minipage}
}
\end{document}
